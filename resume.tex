\documentclass{resume}
\usepackage[left=0.6in,top=0.6in,right=0.6in,bottom=0.6in]{geometry}
\usepackage[svgnames]{xcolor}
\usepackage{hyperref}
\hypersetup{%
  colorlinks=true,          % hyperlinks
  allcolors=DarkRed,%
  pdfborderstyle={/S/U/W 1} % border style will be underline of width 1pt
}

\usepackage{vwcol}
%\usepackage{natbib}
\usepackage{cfr-lm}         % make old style numbers default
\usepackage{graphicx}

\begin{document}

%%%%%%%%%%%%%%%%%%%%%%%%%%%%%%%%%%%%%%%%%%%%%%%% TITLE SECTION
\begin{center}
	\name{Alexandre Naud}
	\contact
	%{}
  {\href{https://scholar.google.com/citations?user=hD2L0ScAAAAJ}{Scholar}}
	%{\href{mailto://alex@naud.xyz}{alex@naud.xyz}}
	{\href{https://orcid.org/0000-0003-4559-2110}{ORCID}}
\end{center}

%%%%%%%%%%%%%%%%%%%%%%%%%%%%%%%%%%%%%%%%%%%%%%%% EDUCATION 
\section{Education} 
\begin{content}
    {\bf ESPUM (Univ. Montreal) | CNRS (Univ. Strasbourg)}  \hfill {\bf 2014 --2021} \\
    {Ph.D. Public Health} \hfill
    {\em Excellent mention}

    {\bf University of Montreal} \hfill {\bf 2013 -- 2014} \\ 
	{Master in Community Health} \hfill 
	{\em Fast tracking to doctorate}

    {\bf University of Montreal} \hfill {\bf 2009 -- 2013} \\ 
    {Minor in Cognitive Science {\em\&} Major in Biology} \hfill 
    {\em Dean's List}
    
    % \vspace{-.25\baselineskip}
    % {\em Honours:} If you'd rather list your awards here than in a separate section later, simply write the awards here and comment out that section (and vice versa).
\sectionlineskip
\end{content}

%%%%%%%%%%%%%%%%%%%%%%%%%%%%%%%%%%%%%%%%%%%%%%%% RESEARCH EXPERIENCE
\section{Experience}
\begin{content}

    \begin{position}{CReSP}{2023 -- present}{Post Doctoral Researcher}{}{}
    \item Methodological research on quantitative methods used to assess collective knowledge.
    \item Applying fuzzy cognitive mapping to understand the causal processess around urban gentrification.
    \end{position}
    \vspace{-.1\baselineskip}

    \begin{position}{Polygon Research}{2019 -- 2023}{Research Professionnal | Developer}{}{}
    \item Developing analytical backend for social science research tools.
    \item Building interactive dashboards for data exploration and visualisations.
    \item Proofreading contents for academic readers.
    \end{position}
    
    \begin{position}{nodata}{2020 -- present}{Data Scientist}{}{}
    	\item Generating and analyzing co-authorship networks of academic researchers.
    	\item Analysing geolocated social network data.
    	\item Building retrieval augmented generation frameworks for academic litterature.
    \end{position}

\sectionlineskip
\end{content}
\vspace{-1\medskipamount}

%%%%%%%%%%%%%%%%%%%%%%%%%%%%%%%%%%%%%%%%%%%%%%%%% AWARDS
\section{Grants} 
\begin{content}
	%DOES NOT WORK' NEED TO BE FIXED
	%\prize[s]{Partenaria}{2023-2024}{has its description on the same line and the option is toggled with \texttt{[s]}.}
	\prize{Partenaria \textpl{-- Postdoc. Grant Program}}{2023 -- 2024}{}
	\prize{Mitacs \textpl{-- Accelaration Program}}{2021 -- 2022}{}
	\prize{PDI \textpl{-- International Doctoral Training}}{2016 -- 2019}{}
	\prize{FRSQ \textpl{-- Ph.D. Degree Funding}}{2015 -- 2018}{}
\end{content}

%%%%%%%%%%%%%%%%%%%%%%%%%%%%%%%%%%%%%%%%%%%%%%%%%% ARTICLE
\section{Journal articles}
\begin{content}
	
	\publi
	{Sueur, C., Fancello, G., \textbf{Naud, A.}, Kestens, Y., Chaix, B. The Complexity of Social Networks in Healthy Aging: Novel Metrics and Their Associations with Psychological Well-Being. Peer Community Journal, Volume 4 (2024), article no. e23. 
	\href{https://doi.org/10.24072/pcjournal.388}{https://doi.org/10.24072/pcjournal.388}}
	
	\publi
	{Collonnaz M., Teodora Riglea R., Kalubi J., O'loughlin J., \textbf{Naud, A.}, Kestens Y., Agrinier Nelly., Minary L. (2022). Social network analysis to study health behaviours in adolescents: A systematic review of methods. Social Science {\em\&} Medicine, 315, 115519. \href{https://doi.org/10.1016/j.socscimed.2022.115519}{https://doi.org/10.1016/j.socscimed.2022.115519}}

	\publi
	{Sueur C., Quque M., \textbf{Naud, A.}, Bergouignan A., Criscuolo F. (2021). Social capital: an independent dimension of healthy ageing. EcoEvoRxiv. \href{https://doi.org/10.32942/osf.io/bg94n}{https://doi.org/10.32942/osf.io/bg94n}}

	\publi
	{\textbf{Naud, A.}, Sueur C., Chaix B., Kestens Y. (2020). Combining social network and activity space data for health research: tools and methods. Health and Place, 66, ISSN: 102454. \href{https://doi.org/10.1016/j.healthplace.2020.102454}{https://doi.org/10.1016/j.healthplace.2020.102454}}

	\publi
	{Kestens Y., Wasfi R., \textbf{Naud A.}, and Chaix B. (2017). “Contextualizing Context”: Reconciling Environmental Exposures, Social Networks, and Location Preferences in Health Research. Current environmental health reports, 4(1), 51–60. ISSN: 2196-5412. \href{https://doi.org/10.1007/s40572-017-0121-8}{https://doi.org/10.1007/s40572-017-0121-8}}

	\publi
	{Kestens Y., Chaix B., Gerber P., Desprès M., Gauvin L., Klein O., Klein,S., Köppen B., Lord S., \textbf{Naud A.}, Patte M., Payette H., Richard L., Rondier P., Shareck M., Sueur C., Thierry B., Vallée J. and Wasfi R. (2016). Understanding the role of contrasting urban contexts in healthy aging: an international cohort study using wearable sensor devices (the CURHA study protocol). BMC geriatrics, 16(1), 1. ISSN: 1471-2318. \href{https://doi.org/10.1186/s12877-016-0273-7}{https://doi.org/10.1186/s12877-016-0273-7}}

	\publi
	{\textbf{Naud, A.}, Chailleux, E., Kestens, Y., Bret, C., Desjardins, D., Petit, O., Barthélémy N. and Sueur, C. (2016). Relations between Spatial Distribution, Social Affiliations and Dominance Hierarchy in a Semi-Free Mandrill Population. Frontiers in psychology, 7. ISSN: 1664-1078.
	\href{https://doi.org/10.3389/fpsyg.2016.00612}{https://doi.org/10.3389/fpsyg.2016.00612}} 

	%\publi
	%{Kestens Y., \textbf{Naud A.}, Steinmetz-Wood M., Vallée J. (2014) What is the importance of postal code for health research? Replied: Canada %Post community mailboxes: Implication for health research. Canadian Journal of Public Health. ISSN: 0008-4263. %\href{https://doi.org/10.17269/cjph.105.4884}{https://doi.org/10.17269/cjph.105.4884}}

	\publi
	{Desjardins D., Nissim W. G., Pitre F. E., \textbf{Naud A.}, and Labrecque M. (2014). Distribution patterns of spontaneous vegetation and pollution at a former decantation basin in southern Québec, Canada. Ecological Engineering, 64, 385–390. ISSN: 0925-8574. \href{https://doi.org/10.1016/j.ecoleng.2014.01.003}{https://doi.org/10.1016/j.ecoleng.2014.01.003}}
\end{content}

%%%%%%%%%%%%%%%%%%%%%%%%%%%%%%%%%%%%%%%%%%%%%%%%% CHAPTERS
\section{Thesis and chapter}
\begin{content}
	
	\publi
	{\textbf{Naud A.} Réseau social et espace d’activité : dynamique socio-spatiale et bien-être émotionnel chez les aînés. Université de Montréal {\em\&} Université de Strasbourg. 2022. \href{https://papyrus.bib.umontreal.ca/xmlui/handle/1866/28476}{https://papyrus.bib.umontreal.ca/xmlui/handle/1866/28476}}
	
	\publi
	{Cantinotti M., \textbf{Naud A.} and Kestens Y. (2016). Quand les idées font réseau : combiner l’analyse de réseau et la cartographie conceptuelle. White D., Genest G.B (Eds.). La santé en réseaux : Explorations des approches relationnelles dans la recherche sociale au Québec. Presse de l’Université du Québec.}

\end{content}

%%%%%%%%%%%%%%%%%%%%%%%%%%%%%%%%%%%%%%%%%%%%%%%%% CHAPTERS
\section{Presentations}
\begin{content}
	
  	\publi
	{\textbf{Naud A.}, Moullec G., Thierry B., Perchoux C., Gerber P., Chaix B., Kestens Y. (2023) A new R package to derive activity space metrics: Illustration with map-based questionnaire and well-being measures from Canadian COHESION cohort data in Canada and Montreal. DKG. 20th of September. Frankfurt, Germany Oral communication \textit{Oral comunication}}


	\publi
	{\textbf{Naud A.}, Kestens Y., Sueur C. (2020) Structural properties of very old adults’ socio-spatial networks. Sunbelt. 5th of June. Paris,
	France. Oral communication - CANCELED COVID-19 \textit{Oral comunication}}


	\publi
	{\textbf{Naud A.}, Kestens Y., Sueur C. (2017) Combining activity space and social networks in research on healthy aging: Describing the socio-spatial environment of older adults residing in Québec, Canada. 17th  International Medical Geography Symposium .4th of July. Angers, France. \textit{Oral comunication}}


	\publi
	{\textbf{Naud A.}, Kestens Y., Sueur C. (2016) Social networks within the built environment: Implication or health aging. International Union for Health Promotion and Education. 25th of May. Curitiba, Brazil. \textit{Oral communication}}
 

	\publi
	{\textbf{Naud A.}, Kestens Y., Sueur C. (2015) A tool to measure social networks and activity location. 16th International Medical Geography Symposium. 7th of July. Vancouver, Canada. \textit{Oral communication}}


	\publi
	{\textbf{Naud A.} (2015) Understanding the role of contrasting urban contexts on social networks and healthy aging. Colloque de l’association étudiante de l’école de santé publique de l’Université de Montréal. 18th of February. Montréal, Canada. \textit{Oral communication} }

	
\end{content}

%%%%%%%%%%%%%%%%%%%%%%%%%%%%%%%%%%%%%%%%%%%%%%%%% PROJECTS
\section{Projects \textbf{\em\&} Extracurriculars} 
\begin{content}
    
    \begin{position}{activspace}{2023 -- present}{Developer}{\href{https://www.github.com/alxn0/activspace/}{GitHub}}{}
	\item R library to analyze nested activity locations data.
	\end{position}  
    
    \begin{position}{AREC}{2023 -- present}{Volunteer}{\href{https://asso-arec.fr/}{asso-arec.fr}}{}
  		\item Activist association and organic produce market.
  	\end{position}  
   
    \begin{position}{CRCHUM - CNRS}{2017 -- 2019}{Organizer}{}{}
        \item Seminars on statistical programming in R.
    \end{position}
    
    \begin{position}{SOS - IN2P3}{2018}{Participant}{}{}
    	\item Summerschool on machine learning and neural network.
    \end{position}
    
    %{\bf Very Simple Project} \enspace {\href{https://alice.ship/lorentz}{\texttt{alice.ship/lorentz}} 
	%{\em A really simple entry with inline description} }
    %\hfill {\bf 2018 -- present}

\sectionlineskip    
\end{content}

%%%%%%%%%%%%%%%%%%%%%%%%%%%%%%%%%%%%%%%%%%%%%%%%% TECHNICAL SKILLS
\section{Technical Skills}
\begin{content}
    \begin{tabular}{ @{} >{\bf}l @{\hspace{6ex}} l }
        Programming Languages & Python, R, Bash, {\em \&} \LaTeX\ \\
        Data analysis & Regression models, Clustering, Dimensionnality reduction \\
                      & Network {\em \&} Geospatial analysis  \\
        Languages & French (Native) {\em \&} English (speak fluently)
    \end{tabular}
\sectionlineskip
\end{content}

%%%%%%%%%%%%%%%%%%%%%%%%%%%%%%%%%%%%%%%%%%%%%%%%% COURSE WORK
%\section{Selected Course Work}
%\begin{tabular}{l l l}
%        \qquad {\bf Mathematics} & \qquad {\bf Physics} & \qquad {\bf Computer Science} \\[.5\smallskipamount]
        
%        Analysis Sequence & Classical {\em\&} Quantum Mechanics & H. Algorithms {\em \&} Data Structures \\ 
%        Complex Analysis & Thermal {\em \&} Statistical Physics & Mathematical Foundations of Machine Learning \\
%        Abstract  Algebra & General Relativity & Reinforcement Learning \\
%        Cryptography &  Quantum Field Theory & Probabilistic Analysis of Algorithms 
%\end{tabular}

\end{document}
